\lecture{4}
\title{Intensional Equality}
\date{31 January 2014}
\maketitle

\newcommand\eqa{\mathrel{=_\alpha}}
\newcommand\eqb{\mathrel{=_\beta}}
\newcommand\eqaobs{\mathrel{(\eqa)_{\mathrm{obs}}}}
\newcommand\eqbobs{\mathrel{(\eqb)_{\mathrm{obs}}}}

%%%%%%%%%%%%%%%%%%%%%%%%

\section{$\beta$-Equality}

Let us call two $\lambda$-terms $e_1$ and $e_2$ \emph{$\alpha$-equivalent} and write $e_1\eqa e_2$ if they are interconvertible by simply renaming bound variables. Let us call two $\lambda$-terms $e_1$ and $e_2$ \emph{$\beta$-equivalent} and write $e_1\eqb e_2$ if either (i) they have a common normal form up to $\alpha$-equivalence, or (ii) neither has a normal form; that is, either (i) they both converge to $\alpha$-equivalent normal forms under some sequence of $\beta$-reductions, or (ii) neither converges under any sequence of $\beta$-reductions. By confluence, $\eqb$ is an equivalence relation.

\begin{theorem}
Allowing arbitrary $\beta$-reductions, the following are equivalent:
\begin{enumerate}
\renewcommand\labelenumi{{\upshape(\roman{enumi})}}
\item
$e_1 \eqaobs e_2$.
\item
$e_2 \eqbobs e_2$.
\item
For all contexts $C\hole$, $C\holed{e_1}\eqb C\holed{e_2}$.
\end{enumerate}
\end{theorem}
\begin{proof}
Recall that for any relation $\equiv$ on values, $e_1\eqobs e_2$ if
\begin{itemize}
\item
$e_1\Downarrow$ iff $e_2\Downarrow$; and
\item
if $e_1\Downarrow v_1$ and $e_2\Downarrow v_2$, then $v_1\equiv v_2$.
\end{itemize}
The equivalence of (i) and (ii) is immediate from the definition, since $\eqa$ and $\eqb$ agree on values.
The equivalence of (ii) and (iii) is simply the definition of $\eqbobs$.
\end{proof}

\begin{theorem}
${\eqaobs}$ is a fixpoint of the monotone map ${\equiv}\mapsto{\eqobs}$ on values.
\end{theorem}
\begin{proof}
It follows immediately from the definitions that ${\eqaobs}$, ${\eqbobs}$, ${\eqa}$, and $\eqb$ agree on values.
\end{proof}

\subsection{Contexts and Observational Equivalence}

Another approach to the problem of defining equivalence is to say that
two terms are equivalent if they behave indistinguishably in any possible context.
But what do we mean by ``behave indistinguishably''?

For simplicity, let us assume that we are working with an evaluation strategy such as CBV or CBN that is _deterministic_, which means that there is at most one next $\beta$-reduction that can be performed. We say that a term $e$ _terminates_ or _converges_ if there is a finite sequence of reductions
\[
e\ \to\ e'\ \to\ e''\ \to\ \cdots\ \to\ v
\]
leading to a value $v$. We write $e\Downarrow v$ when this happens, and we write $e\Downarrow$ when $e\Downarrow v$ for some $v$. The other possibility is that it keeps on reducing forever without ever arriving at a value. When this happens, we say that $e$ _diverges_ and write $e \Uparrow$. Because we have assumed that we are using a deterministic evaluation strategy, exactly one of these two cases will occur.

With CBN or CBV, there are infinitely many divergent terms. One example is $\Omega$, which was defined in the last lecture. 
We might consider all divergent terms equivalent, since none of them produce a value.

While we may not have a precise definition of extensional equivalence yet, we can postulate a desirable property: two equivalent terms, when placed in the same context, should either both diverge or both converge and give indistinguishable values. Here a _context_ is any term $C\hole$ with a single occurrence of a distinguished special variable, called the _hole_, and $C\holed e$ denotes the context $C\hole$ with the hole replaced by the term $e$. This notion of equivalence is called \emph{observational equivalence}.

More formally, suppose we already have a notion of equivalence $\equiv$ on values.
Then we will say that two terms are \emph{observationally equivalent} (with respect to $\equiv$) and write $e_1\eqobs e_2$ if for all contexts $C\hole$,
\begin{itemize}
\item
$C\holed{e_1} \Downarrow$ iff $C\holed{e_2} \Downarrow$; and
\item
if $C\holed{e_1} \Downarrow v_1$ and $C\holed{e_2} \Downarrow v_2$, then $v_1\equiv v_2$.
\end{itemize}
In other words, either both $C\holed{e_1}$ and $C\holed{e_2}$ diverge, or both converge and produce equivalent values.

Note that on values themselves, equivalence is not necessarily the same as observational equivalence. Certainly two values that are observationally equivalent are equivalent in the sense of $\equiv$, because we could put them in the trivial context consisting of just the hole. However, the converse is not true: we could easily have values that are equivalent in the sense of $\equiv$ but not observationally equivalent. Is it possible to have $\eqobs$ and $\equiv$ coincide on values? In other words,
does there exist a \emph{fixpoint} of the transformation ${\equiv}\mapsto{\eqobs}$? If so, is it unique? Even if not, is there a reasonable choice for the definition of extensional equivalence?

The answers to these questions lie in the following facts, none of which are difficult to prove.
We leave them as exercises.
\begin{lemma}
\label{lem:eqobs}
Let $\equiv$ be an arbitrary equivalence relation on values.
\begin{enumerate}
\renewcommand\labelenumi{\upshape(\roman{enumi})}
\item
The relation $\eqobs$ is an equivalence relation on terms.
\item
Restricted to values, $\eqobs$ \emph{refines} $\equiv$; that is, viewed as sets of ordered pairs, $\eqobs$ restricted to values is a subset of $\equiv$. Thus for any values $v_1$ and $v_2$, if $v_1\eqobs v_2$, then $v_1\equiv v_2$.
\item
If $e_1\eqobs e_2$, then for all contexts $C\hole$, $C\holed{e_1}\Downarrow$ iff $C\holed{e_2}\Downarrow$.
\item
The transformation ${\equiv}\mapsto{\eqobs}$ is \emph{monotone} with respect to the refinement relation. That is, if $\equiv^1$ refines $\equiv^2$, then $\equiv^1_{\mathrm{obs}}$ refines $\equiv^2_{\mathrm{obs}}$.
\end{enumerate}
\end{lemma}
%\begin{proof}
%Miscellaneous Exercise \ref{}.
%\end{proof}
Now we can see that there are several fixpoints of the transformation ${\equiv}\mapsto{\eqobs}$; the identity relation and the relation of $\alpha$-equivalence, for two. This follows from Lemma \ref{lem:eqobs}(i) and (ii). For CBV and CBN, there is also a \emph{coarsest} one that is refined by every other fixpoint: define
\begin{align*}
e_1\eqdown e_2\ \ &\Iffdef\ \ \text{for all contexts } C\hole,\ C\holed{e_1}\Downarrow \text{ iff } C\holed{e_2}\Downarrow.
\end{align*}
\begin{theorem}
\label{thm:obs}
For CBV and CBN, the relation $\eqdown$ is a fixpoint of the transformation ${\equiv}\mapsto{\eqobs}$;
that is, ${\eqdown}=({\eqdown)_{\mathrm{obs}}}$. Moreover, it is the coarsest such fixpoint.
\end{theorem}
%\begin{proof}
%Miscellaneous Exercise \ref{}.
%\end{proof}
The relation $\eqdown$ may be a reasonable candidate for extensional equivalence. 
By definition, to check that $e_1$ and $e_2$ are observationally equivalent, it is enough to check that $e_1$ and $e_2$ both converge or both diverge in any context; it is unnecessary to compare the resulting values in the case of convergence. This is because if the values are not equivalent, one can devise a context in which one converges and the other diverges.

%%%%%%%%%%%%%%%%
