\lecture{20}
\title{Soundness of the Typing Rules}
\date{8 April 2016}
\maketitle

\section{Soundness from the Operational Perspective}

We have seen that we can write useful typing rules, but how do we know we got them right?
This depends on what the type system is meant to accomplish.
The traditional application is avoiding runtime type errors.
We say that a type system is _sound_ if well-typed programs do not incur
runtime type errors; that is, they do not get stuck when evaluated
according to the operational semantics:
\begin{center}
\emph{The typing rules are sound} $\Iffdef$ \emph{no well-typed program gets stuck.}
\end{center}

To be more precise, let us call $e$ _irreducible_ and write \Irred e\
if there is no reduction possible on $e$. All values of $\lambda^\to$
are irreducible. If $e$ is irreducible but is not a value, then $e$ is
said to be _stuck_. We wish to show
\begin{theorem}[Operational Soundness]
\label{lem:operationalsoundness}\
If $\judge{}e\tau$ and $e \stackrel*\to e'$ and $\Irred{e'}$, then $e'\in\Val$ and $\judge{}{e'}\tau$.
\end{theorem}
We will prove this in two steps using the following two lemmas:
\begin{lemma}[Type Preservation]
\label{lem:typepreservation}\
If $\Gjudge e\tau$ and $e \to e'$, then $\Gjudge{e'}\tau$.
\end{lemma}
\begin{lemma}[Progress]
\label{lem:progress}\
If $\judge{}e\tau$ and $\Irred e$, then $e\in\Val$.
\end{lemma}
The type preservation lemma (Lemma \ref{lem:typepreservation}) says that as we evaluate a program, its type is preserved at each step. The progress lemma (Lemma \ref{lem:progress}) says that every program is either a value or can be stepped to another program (and by type preservation, this will be of the same type).

Operational soundness follows easily from these two lemmas. Type preservation
says every step preserves the type, so we use
induction on the number of steps taken in $e \stackrel*\to e'$ to show
that $e'$ must have the same type as $e$. Then progress can
be applied to $e'$ to show that the evaluation is not stuck there.
We will now set out to prove these two lemmas.

\section{Proof of the Type Preservation Lemma}

Assuming that $\Gjudge e\tau$ and $e\to e'$, we wish to show that $\Gjudge{e'}\tau$.
We will do this by induction on the small-step operational semantics rules.

If $e\to e'$, there are three cases corresponding to the three evaluation rules:
\[
\begin{array}{c@{\hspace{1cm}}c@{\hspace{1cm}}c}
\infer[\mbox{(L)}]{e_0\,e_1 \to e'_0\,e_1}{e_0 \to e'_0} &
\infer[\mbox{(R)}]{v\,e \to v\,e'}{e \to e'} &
\infer[(\beta)]{(\lam{\type x\sigma}{e})\,v \to \subst evx}{}
\end{array}
\]

\begin{itemize}
\item Case (L): $e_0\,e_1 \to e'_0\,e_1$.

Because we have a typing derivation for $e_0\,e_1$, we know
that there are typing derivations for $e_0$ and $e_1$ too. We must
have $\Gjudge{e_0}{\sigma\to\tau}$ and $\Gjudge{e_1}\sigma$
for some type $\sigma$. By the induction hypothesis,
the reduction $e_0 \to e_0'$ also preserves type, so
$\Gjudge{e_0'}{\sigma\to\tau}$. Applying the typing rule for
applications, we have that $\Gjudge{e_0'\,e_1}\tau$.

\item Case (R): $v\,e \to v\,e'$.

This case is symmetrical to case (L). In this case it is the right-hand
subexpression making the step.

\item Case ($\beta$): $(\lam{\type x\sigma}{e})\,v \to \subst evx$.

The typing derivation of $\Gjudge{(\lam{\type x\sigma}{e})\,v}\tau$ must look like this:

\[
  \infer{
     \Gjudge{(\lam{\type x\sigma}{e})\,v}\tau
     }
     {
     \infer {
        \Gjudge{(\lam{\type x\sigma}{e})}{\sigma\to\tau}
        }
        {
        \Gamma,\,\type x\sigma\judge{}{e}\tau
        }
     &
     \Gjudge v\sigma
     }
\]

We want to show that $\Gjudge{\subst evx}\tau$ using the facts
$\judge{\Gamma,\,\type x\sigma}e\tau$ and $\judge{}v\sigma$.
Our induction hypothesis does not help us here; we need to
prove this separately. It follows as a special case of the
substitution lemma below, which captures the type
preservation of $\beta$-reduction.
\end{itemize}

\section{The Substitution Lemma}

\begin{lemma}[Substitution Lemma]\
$\judge{}v\sigma\ \Rightarrow\ (\Gamma,\,\type x\sigma\judge{}e\tau\ \Leftrightarrow\ \Gjudge{\subst evx}\tau)$.
\end{lemma}
We will prove this by structural induction on $e$.

\paragraph{Case 1} $x\notin\FV e$.

This case covers the base cases $e\in\{n, "true", "false", "null"\}$ and $e = y\neq x$ and $\lambda$-abstractions $\lam{\type x\rho}e$ that bind $x$. In this case the substitution has no effect and any binding of $x$ in the type environment $\Gamma$ is irrelevant, thus the lemma reduces to the trivial statement
\begin{align*}
\judge{}v\sigma\ &\Rightarrow\ (\Gjudge e\tau\ \Leftrightarrow\ \Gjudge e\tau).
\end{align*}

\paragraph{Case 2} $e = x$.

In this case the lemma reduces to
\begin{align*}
\judge{}v\sigma\ &\Rightarrow\ (\Gamma,\,\type x\sigma\judge{}x\tau\ \Leftrightarrow\ \Gjudge v\tau),
\end{align*}
since $\subst xvx=v$. Since $v$ is closed, the type environment $\Gamma$ is irrelevant, so the statement further reduces to
\begin{align*}
\judge{}v\sigma\ &\Rightarrow\ (\type x\sigma\judge{}x\tau\ \Leftrightarrow\ \judge{}v\tau).
\end{align*}
Since types are unique, both sides of the double implication say that $\sigma=\tau$, so again the lemma reduces to a tautology.

\paragraph{Case 3} $e = e_0\,e_1$.

Suppose $\judge{}v\sigma$.
\reasoning{\exists\sigma\ \Gjudge{\subst{e_0}vx}\sigma\to\tau\ \wedge\ \Gjudge{\subst{e_1}vx}\sigma}
\begin{align*}
\Gamma,\,\type x\sigma\judge{}{e_0\,e_1}\tau\
&\Leftrightarrow\ \reason{\exists\rho\ \Gamma,\,\type x\sigma\judge{}{e_0}\rho\to\tau\ \wedge\ \Gamma,\,\type x\sigma\judge{}{e_1}\rho}{typing rule for applications}\\
&\Leftrightarrow\ \reason{\exists\rho\ \Gjudge{\subst{e_0}vx}\rho\to\tau\ \wedge\ \Gjudge{\subst{e_1}vx}\rho}{induction hypothesis}\\
&\Leftrightarrow\ \reason{\Gjudge{(\subst{e_0}vx)\,(\subst{e_1}vx)}\tau}{typing rule for applications}\\
&\Leftrightarrow\ \reason{\Gjudge{\subst{(e_0\,e_1)}vx}\tau}{definition of substitution.}
\end{align*}

\paragraph{Case 4} $e = \lam{\type y\rho}{e'}$, where $y\neq x$ (the case $y=x$ was covered in Case 1).

Suppose $\judge{}v\sigma$. The type of $\lam{\type y\rho}{e'}$, if it exists, must be $\rho\to\tau$ for some $\tau$. Similarly, the type of $\subst{(\lam{\type y\rho}{e'})}vx=\lam{\type y\rho}{(\subst{e'}vx)}$, if it exists, must be $\rho\to\tau'$ for some $\tau'$.
\reasoning{\Gjudge{\subst{(\lam{\type y\rho}{e'})}vx}{\rho\to\tau}}
\begin{align*}
\judge{\Gamma,\,\type x\sigma}{(\lam{\type y\rho}{e'})}{\rho\to\tau}\
&\Leftrightarrow\ \reason{\judge{\Gamma,\,\type x\sigma,\,\type y\rho}{e'}\tau}{typing rule for abstractions}\\
&\Leftrightarrow\ \reason{\judge{\Gamma,\,\type y\rho,\,\type x\sigma}{e'}\tau}{exchange}\\
&\Leftrightarrow\ \reason{\judge{\Gamma,\,\type y\rho}{\subst{e'}vx}\tau}{induction hypothesis}\\
&\Leftrightarrow\ \reason{\Gjudge{\lam{\type y\rho}{(\subst{e'}vx)}}{\rho\to\tau}}{typing rule for abstractions}\\
&\Leftrightarrow\ \reason{\Gjudge{\subst{(\lam{\type y\rho}{e'})}vx}{\rho\to\tau}}{definition of substitution.}
\end{align*}

\section{Proof of the Progress Lemma}

To finish the proof of soundness, it remains to prove the progress lemma. Recall that this lemma states
\begin{align*}
\judge{}e\tau\ \wedge\ \Irred e\ &\Rightarrow\ e\in\Val,
\end{align*}
or equivalently,
\begin{align*}
\judge{}e\tau\ \wedge\ e\notin\Val\ &\Rightarrow\ \exists e'\ e\to e'.
\end{align*}
In other words, we cannot get stuck when evaluating a well-typed expression.

We prove the progress lemma using structural induction on $e$.
Recall the definition of a term in $\lambda^\to$:
\begin{align*}
e\ &::=\ b \bnf x \bnf \lam{\type x\tau}e \bnf e_0\,e_1,
\end{align*}
where $b$ denotes a constant. This gives four cases:

\paragraph{Case 1} $e=b$.

Since $b\in\Val$, we are done.

\paragraph{Case 2} $e=x$.

This case is impossible, because we cannot assign a type to $x$ if the type environment is empty.

\paragraph{Case 3} $e=\lam{\type x\sigma}{e'}$.

This case requires another lemma:
\begin{lemma}
If $\Gjudge e\tau$ then $\FV e\subseteq\dom\Gamma$.
\end{lemma}
We leave the proof as an exercise. Since $\judge{}e\tau$, it follows that $e$ is closed, therefore is a value.

\paragraph{Case 4} $e=e_0\,e_1$.

We cannot have a value of this form, so the statement of the lemma reduces to
\begin{align*}
\judge{}{(e_0\,e_1)}\tau\ &\Rightarrow\ \exists e'\ (e_0\,e_1)\to e'.
\end{align*}
In any type derivation of $\judge{}{(e_0\,e_1)}\tau$, the last step must have the form
\[
\infer{\judge{}{(e_0\,e_1)}\tau}{\judge{}{e_0}{\sigma\to\tau} & \judge{}{e_1}\sigma}
\]
for some type $\sigma$. By the induction hypothesis, either $e_0\in\Val$ or $\exists e_0'\ e_0\to e_0'$, and either $e_1\in\Val$ or $\exists e_1'\ e_1\to e_1'$. This gives three possibilities:
\begin{itemize}
\item If $e_0$ is not a value, then by the induction hypothesis there $\exists e_0'\ e_0\to e_0'$, therefore
\[
\infer{e_0\,e_1\to e_0'\,e_1}{e_0\to e_0'},
\]
so $e=e_0\,e_1$ can be further reduced.

\item If $e_0$ is a value $v$ but $e_1$ is not a value, then by the induction hypothesis $\exists e_1'\ e_1\to e_1'$, and we have
\[
\infer{v\,e_1\to v\,e_1'}{e_1\to e_1'},
\]
so $e=v\,e_1$ can be further reduced.
\item
If both $e_0$ and $e_1$ are values, then since $e_0$ is a value with an arrow type $\sigma\to\tau$, it has to be an abstraction, say $e_0=\lam{\type x\sigma}{e'}$, and $e_1$ is some value $v$ of type $\sigma$. Then
\begin{align*}
e\ &=\ (\lam{\type x\sigma}{e'})\,v \to \subst{e'}vx,
\end{align*}
so $e$ can be further reduced.
\end{itemize}
This completes the proof.
