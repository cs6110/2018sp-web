\lecture{10}
\title{Adequacy}
\date{22 February 2016}
\maketitle

Both the CBV and CBN $\lambda$-calculus are \emph{deterministic} reduction strategies in the sense that there is at most one reduction that is enabled in any term. When an expression $e$ in a language is evaluated in a deterministic system, one of three things can happen:
\begin{enumerate}
\item
There exists an infinite sequence of expressions $e_1,e_2,\ldots$ such that $e\stepsone e_1\stepsone e_2\stepsone\ldots$~. In this case, we write $e\Uparrow$ and say that $e$ \emph{diverges}.
\item
The expression $e$ produces a value $v$ in zero or more steps. In this case we say that $e$ \emph{converges} to the value $v$ and write $e\Downarrow v$.
\item
The computation converges to a non-value. When this happens, we say the computation is \emph{stuck}.\footnote{This cannot happen with our CBN-to-CBV translation, but we will see some examples soon enough.}
\end{enumerate}

A semantic translation is \emph{adequate} if these three behaviors in the source system are accurately reflected in the target system, and vice versa. This relationship is illustrated in the following diagram:
%\begin{center}
%\begin{picture}(100,50)
%%\thicklines
%\put(0,0){\makebox(0,0)[c]{$\SB e$}}
%\put(0,50){\makebox(0,0)[c]{$e$}}
%\put(53,0){\makebox(0,0)[c]{$v'$}}
%\put(80,50){\makebox(0,0)[c]{$v$}}
%\put(80,0){\makebox(0,0)[c]{$\SB v$}}
%\put(-3,25){\makebox(0,0)[r]{$\SB\cdot$}}
%\put(83,25){\makebox(0,0)[l]{$\SB\cdot$}}
%\put(80,44){\vector(0,-1){36}}
%\put(0,44){\vector(0,-1){36}}
%\put(8,0){\vector(1,0){37}}
%\put(8,50){\vector(1,0){65}}
%\put(26,3){\makebox(0,0)[b]{$*$}}
%\put(40,53){\makebox(0,0)[b]{$*$}}
%\put(65,0){\makebox(0,0)[c]{$\approx$}}
%\end{picture}
%\end{center}
\begin{center}
\begin{tikzpicture}[->, >=stealth', node distance=24mm, auto]
\small
 \node (NW) {$e$};
 \node (NE) [right of=NW] {$v$};
 \node (SW) [below of=NW, node distance=16mm] {$\SB e$};
 \node (SE) [below of=NE, node distance=16mm, xshift=-3.2mm] {$v'\approx\SB v$};
 \path (NW) edge node {$_{\textstyle*}$} (NE);
 \path (NW) edge node [swap] {$\SB\cdot$} (SW);
 \path (SW) edge node {$_{\textstyle*}$} (SE);
 \path (NE) edge node {$\SB\cdot$} ($(SE.north)!0.5!(SE.north east)$);
\end{tikzpicture}
\end{center}
If $e$ converges to a value $v$ in the source language, then $\SB e$ must converge to some value $v'$ that is equivalent (e.g.~$\beta$-equivalent) to $\SB v$ in the target language, and vice-versa. This is formally stated as two properties, \emph{soundness} and \emph{completeness}. For our CBN-to-CBV translation, these properties take the following form:

\textsf{Soundness:}
\begin{enumerate}
\romanize
\item
$\SB e\Downarrow_{\mathrm{cbv}} t\ \Imp\ \exists v\ t\approx\SB v \wedge e\Downarrow_{\mathrm{cbn}} v$
\item
$\SB e\Uparrow_{\mathrm{cbv}}\ \Imp\ e\Uparrow_{\mathrm{cbn}}$
\end{enumerate}
\textsf{Completeness:}
\begin{enumerate}
\romanize
\item
$e\Downarrow_{\mathrm{cbn}} v\ \Imp\ \exists t\ t\approx\SB v \wedge \SB e\Downarrow_{\mathrm{cbv}} t$
\item
$e\Uparrow_{\mathrm{cbn}}\ \Imp\ \SB e\Uparrow_{\mathrm{cbv}}$.
\end{enumerate}
%
%\reasoning{\exists v'\ \SB e \goesto *{\mathrm{cbv}} v' \wedge v'\approx\SB v}
%\begin{eqnarray*}
%\SB e \goesto *{\mathrm{cbv}} v' &\Imp& \reason{\exists v\ v'\approx\SB v \wedge e \goesto *{\mathrm{cbn}} v}{(soundness)}\\
%e \goesto *{\mathrm{cbn}} v &\Rightarrow& \reason{\exists v'\ v'\approx\SB v \wedge \SB e \goesto *{\mathrm{cbv}} v'}{(completeness)}
%\end{eqnarray*}
where $\approx$ is some notion of target term equivalence that is preserved by evaluation.

Soundness says that the computation in the CBV domain starting from the image $\SB e$ of a CBN program $e$ accurately simulates the computation starting from $e$ in the CBN domain. Thus if the CBV process terminates in a value, then so must the CBN process, and the values must be related in the sense described formally above; and if the CBV computation diverges, then so does the CBN computation. Completeness says the opposite:\ the computation in the CBN domain starting from $e$ is accurately simulated by the computation in the CBV domain starting from $\SB e$.

\emph{Adequacy} is the combination of soundness and completeness.

\section{Proving Adequacy}

We would like to show that evaluation commutes with our translation
$\SB\cdot$ from CBN to CBV. To do this, we first need a notion of target term
equivalence ($\approx$) that is preserved by evaluation. This is made more
challenging because as evaluation takes place in the target language,
intermediate terms are generated that are not the translation of any
source term. For some translations (but not this one), the reverse may
also happen. Therefore, equivalence needs to allow for some extra
$\beta$-redexes that appear during translation. We can define this equivalence
by structural induction on CBV target terms according to the following rules:
\begin{center}
\AxiomC{$x\approx x$}
\DisplayProof
\quad
\AxiomC{$t \approx t'$}
\UnaryInfC{$\lam xt \approx \lam x{t'}$}
\DisplayProof
\quad
\AxiomC{$t_0 \approx t_0'$}
\AxiomC{$t_1 \approx t_1'$}
\BinaryInfC{$t_0\,t_1 \approx t_0'\,t_1'$}
\DisplayProof
\quad
\AxiomC{$t \approx (\lam zt)\,\ID$, \text{where $z \not\in \FV t$}}
\DisplayProof
\end{center}
Here, $t$ represents target terms, to keep them distinct from source
terms $e$. We also include rules so that the relation $\approx$ is reflexive,
symmetric, and transitive. One can show easily that if two terms are
equivalent with respect to this relation, then they have the same
$\beta$-normal form.

To show adequacy, we show that each evaluation step in the source
term is mirrored by a sequence of evaluation steps in the corresponding target term,
and vice versa.  So we define a correspondence $\apprle$
between source and target terms that is more general than the
translation $\SB\cdot$ and is preserved during evaluation of both source
and target.

We write $e\apprle t$ to mean that CBN term $e$ corresponds to CBV term $t$.
The following proposition captures the idea that CBV evaluation
simulates CBN evaluation at the level of individual steps:
\begin{eqnarray}
e\apprle t\ \wedge\ e\to e' &\Rightarrow& \exists t'\ t\xrightarrow * t'\ \wedge\ e'\apprle t'
\label{eqn:prop}
\end{eqnarray}
This can be visualized as a commutative diagram:
%\begin{center}
%\begin{picture}(100,50)
%\put(0,0){\makebox(0,0)[c]{$t$}}
%\put(0,50){\makebox(0,0)[c]{$e$}}
%\put(50,0){\makebox(0,0)[c]{$t'$}}
%\put(50,50){\makebox(0,0)[c]{$e'$}}
%\put(0,8){\vector(0,1){36}}
%\put(0,8){\vector(0,-1){0}}
%\put(-4,25){\makebox(0,0)[r]{$\apprle$}}
%\put(54,25){\makebox(0,0)[l]{$\apprle$}}
%\put(50,8){\vector(0,1){36}}
%\put(50,8){\vector(0,-1){0}}
%\put(8,0){\vector(1,0){36}}
%\put(8,50){\vector(1,0){36}}
%\put(25,3){\makebox(0,0)[b]{$*$}}
%\put(75,0){\makebox(0,0)[c]{$(\approx\SB{e'})$}}
%\end{picture}
%\end{center}
\begin{center}
\begin{tikzpicture}[->, >=stealth', node distance=24mm, auto]
\small
 \node (NW) {$e$};
 \node (NE) [right of=NW] {$e'$};
 \node (SW) [below of=NW, node distance=16mm] {$t$};
 \node (SE) [below of=NE, node distance=16mm] {$t'$};
 \node (SEi) [right of=SE, node distance=2mm, anchor=west] {$(\approx\SB{e'})$};
 \path (NW) edge node {$1$} (NE);
 \path (NW) edge[-] node [swap] {$\apprle$} (SW);
 \path (SW) edge node {$_{\textstyle*}$} (SE);
 \path (NE) edge[-] node {$\apprle$} (SE);
\end{tikzpicture}
\end{center}
In fact, since in this case the source language cannot get stuck during
evaluation, and both languages have deterministic evaluation,
\eqref{eqn:prop} ensures that evaluation in each language corresponds
to the other.

We define the relation $\apprle$ in such a way that $e \apprle \SB e $. Then,
using \eqref{eqn:prop}, we can show that any trace in the source
language produces a corresponding trace in the target by induction on
the number of source-language steps.

We define the relation $\apprle$ by the following rules:

\begin{equation}
\AxiomC{$x\apprle x\,\ID$}
\DisplayProof
\qquad
\AxiomC{$e\apprle t$}
\UnaryInfC{$\lam xe \apprle \lam xt$}
\DisplayProof
\qquad
\AxiomC{$e_0\apprle t_0$}
\AxiomC{$e_1\apprle t_1$}
\BinaryInfC{$e_0\,e_1 \apprle t_0\,(\lam \, t_1)$}
\DisplayProof
\qquad
\AxiomC{$e\apprle t$}
\UnaryInfC{$e \apprle (\lam\, t)\,\ID$}
\DisplayProof
\label{eqn:sim}
\end{equation}

For simplicity, we ignore the fresh variable that would be used in
the new lambda abstraction in the last two rules.

The first three rules of \eqref{eqn:sim} ensure that a source term corresponds to its translation.
The last rule is different; it takes care of the extra $\beta$-reductions that
may arise during evaluation. Because the right-hand side of the $\apprle$ relation becomes
structurally smaller in this rule's premise, the definition of the relation is still well-founded.
The first three rules are well-founded based on the structure of $e$; the last
is well-founded based on the structure of $t$. If we were proving a more complex translation correct,
we would need more rules like the last rule for other
meaning-preserving target-language reductions.

First, let us warm up by showing that a term corresponds to its translation.

\begin{lemma}
$e \apprle \SB e$.
\label{lem:trans}
\end{lemma}
\begin{proof}
An easy structural induction on $e$.
\begin{itemize}
\item Case $x$: $x \apprle x\,\ID$ by definition.
\item Case $\lam x e'$: We have $\SB e = \lam x \SB{e'} $.
By the induction hypothesis, $e' \apprle \SB{e'} $, so $\lam x e' \apprle \lam x \SB{e'} $
by the second rule of \eqref{eqn:sim}.
\item Case $e_0\,e_1$: We have $\SB e = \SB{e_0}\,(\lam\,{\SB{e_1}})$.
By the induction hypothesis, $e_0 \apprle \SB{e_0}$ and $e_1 \apprle \SB{e_1}$. Therefore
by the third rule of \eqref{eqn:sim}, $e_0\,e_1 \apprle \SB{e_0}\,\SB{e_1}$.
\end{itemize}
\end{proof}

Next, let us show that if $e$ corresponds to $t$, its translation is equivalent to $t$.

\begin{lemma}
$e \apprle t\ \Rightarrow\ \SB e \approx t$.
\label{lem:equiv}
\end{lemma}
\begin{proof}
Induction on the derivation of $e \apprle t$.

\begin{itemize}
\item Case $x \apprle x\,\ID$:\\
This case is trivial: $\SB x = x\,\ID$.

\item Case $\lam x {e'} \apprle \lam x{t'}$ where $e' \apprle t'$:\\
Here, $\SB e = \lam x {\SB{e'} }$.
By the induction hypothesis, $\SB{e'}\approx t'$, therefore
$\lam x{\SB{e'} } \approx \lam x{t'}$ as required.

\item Case $e_0\,e_1 \apprle t_0\,(\lam\,{t_1})$ where $e_0\apprle t_0$ and $e_1\apprle t_1$:\\
Here, $\SB{e_0\,e_1} = \SB{e_0} (\lam\,\SB{e_1})$, and by the induction hypothesis,
$\SB{e_0} \approx t_0$ and $\SB{e_1} \approx t_1$.
From the definition of $\approx$, we have
$\SB{e_0} (\lam\,\SB{e_1}) \approx t_0\,(\lam\,{t_1})$.

\item Case $e \apprle (\lam\,t)\,\ID$ where $e\apprle t$:\\
The induction hypothesis is $\SB e \approx t$. But $t \approx (\lam\,t)\,\ID$, and
$\approx$ is transitive.
\end{itemize}
\end{proof}

Given these definitions, we can prove \eqref{eqn:prop} by induction on
the derivation of $e\apprle t$. We will need two useful lemmas. The first is
a substitution lemma that says substituting corresponding terms into
corresponding terms produces corresponding terms:

\begin{lemma}
$e_1 \apprle t_1 \wedge e_2 \apprle t_2 \Rightarrow \subst{e_2}{e_1}{x} \apprle \subst{t_2}{\lam{}{t_1}}{x}$.
\label{lem:subst}
\end{lemma}
\begin{proof}
We proceed by induction on the derivation of $e_2 \apprle t_2$.
\begin{itemize}
\item Case $x \apprle x\,\ID$:\\
We have $\subst{e_2}{e_1}{x} = e_1$ and
$\subst{t_2}{\lam{}{t_1}}{x} = (\lam{}{t_1})\,\ID$.
By the fourth rule of \eqref{eqn:sim},
we have $e_1 \apprle (\lam{}{t_1})\,\ID$.

\item Case $y \apprle y\,\ID$ where $y\neq x$: \\
This case is trivial, as the substitution has no effect.

\item Case $\lam xe \apprle \lam xt$ where $e\apprle t$: \\
Again, this case is trivial, as the substitutions into $e_2$ and $t_2$ have no effect.
\item Case $\lam ye \apprle \lam yt$ where $e\apprle t$, $x\neq y$:\\
Here 
$\subst{e_2}{e_1}{x} = \lam y{\subst{e}{e_1}{x}}$ and
$\subst{t_2}{\lam{}{t_1}}{x} = \lam y{\subst{t}{\lam{}{t_1}}{x}}$.
Since $e\apprle t$, by the induction hypothesis we have
$\subst{e}{e_1}{x} \apprle \subst{t}{\lam{}{t_1}}{x}$.
Therefore by definition, $\lam y{\subst{e}{e_1}x} \apprle \lam
y{\subst{t}{\lam{}{t_1}}x}$, as required.

\item Case $e\,e' \apprle t\,(\lam{}{t'})$, where $e\apprle t$ and $e'\apprle t'$: \\
We have $\subst{e_2}{e_1}{x} = \subst{e}{e_1}{x}\,\subst{e'}{e_1}{x}$,
and $\subst{t_2}{\lam{}{t_1}}{x} =
\subst{t}{\lam{}{t_1}}{x}\,(\lam{}{\subst{t'}{t_1}{x}})$. From the
induction hypothesis, $\subst{e}{e_1}{x} \apprle \subst{t}{\lam{}{t_1}}x$
and $\subst{e'}{e_1}{x} \apprle \subst{t'}{\lam{}{t_1}}x$.
Therefore, by definition we have
$\subst{e}{e_1}{x}\,\subst{e'}{e_1}{x}
\apprle \subst{t}{\lam{}{t_1}}{x}\,(\lam{}{\subst{t'}{t_1}{x}})$.

\item Case $e_2 \apprle (\lam\, t_2')\,\ID$, where $e_2\apprle t_2'$: \\
We need to show that $\subst{e_2}{e_1}{x} \apprle \subst{((\lam\,t_2')\,\ID)}{\lam\,t_1}{x}$; that is,
$\subst{e_2}{e_1}{x} \apprle ((\lam\, \subst{t_2'}{\lam\,t_1}{x})\,\ID)$.
From the induction hypothesis, we have $\subst{e_2}{e_1}{x} \apprle 
\subst{t_2'}{\lam{}{t_1}}x$. By definition, this means
$\subst{e_2}{e_1}{x} \apprle (\lam\,\subst{t_2'}{\lam{}{t_1}}x)\,\ID$.
\end{itemize}
\end{proof}

The next lemma says that if a value $\lam xe$ corresponds
to a term $t$, then $t$ reduces to a corresponding $\lambda$-term $\lam\,t'$.

\begin{lemma}
\label{lem:value-reduction}
$\lam xe\apprle t \Rightarrow \exists t'\ t \to \lam{x}{t'}\ \wedge\ e \apprle t'$.
\end{lemma}
\begin{proof}
By induction on the derivation of $\lam xe\apprle t$.
\begin{itemize}
\item Case $y\apprle y\,\ID$: Impossible, as $y \ne \lam{x}{e}$.
\item Case $\lam xe \apprle \lam x{t'}$ where $e\apprle t'$: \\
Here, $t = \lam x{t'}$, and the result is immediate.

\item Case $e_0\,e_1 \apprle t_0\,(\lam{}{t_1})$: Impossible, as $e_0\,e_1 \ne \lam{x}{e}$.

\item Case $e_0 \apprle (\lam{}{t_0})\,\ID $, where $e_0\apprle t_0$: \\ In this case
$e_0 = \lam{x}{e}$, and $t =((\lam{}{t_0})\,\ID )$. By the induction
hypothesis, there is some $t'$ such that $t_0 \to 
\lam{x}{t'}$ and $e \apprle t'$. Since $t =((\lam{}{t_0})\,\ID ) \stepsone t_0$ we
have $t \to \lam{x}{t'}$, as required.
\end{itemize}
\end{proof}

We are now ready to prove \eqref{eqn:prop}.

\begin{proof}
By induction on the derivation of $e\apprle t$:
\begin{itemize}
\item Case $x \apprle x\,\ID$:
Vacuously true, as there is no evaluation step $e\stepsone e'$.
\item Case $\lam x e \apprle \lam x t$:
A value: also vacuously true.
\item Case $e_0\,e_1\apprle t_0\,(\lam{}{t_1})$, where $e_0\apprle t_0$ and $e_1\apprle t_1$:\\
We show this by cases on the derivation of $e\stepsone e'$:
\begin{itemize}
\item Case $e_0\,e_1\stepsone e_0'\,e_1$, where $e_0\stepsone e_0'$:\\
By the induction hypothesis, $\exists t_0'\ e_0'\apprle t_0'\wedge t_0\to t_0'$.
It is easy to see that $t_0\,(\lam{}{t_1})\to t_0'\,(\lam{}{t_1})$.
By the third rule of \eqref{eqn:sim}, $e_0'\,e_1 \apprle t_0'\,(\lam{}{t_1})$, as required.

\item Case $(\lam{x}{e_2})\,e_1\stepsone \subst{e_2}{e_1}x$:\\
Here $\lam{x}{e_2} \apprle t_0$ and $e_1\apprle t_1$. 

By Lemma~\ref{lem:value-reduction}, there exists a $t_2$
such that $t_0 \to \lam{x}{t_2}$ and $e_2 \apprle t_2$. Therefore,
we have $t_0\,(\lam \, t_1) \to (\lam{x}{t_2})~(\lam \, t_1)
\stepsone \subst{t_2}{\lam{}{t_1}}x$. But from the substitution lemma
above (Lemma~\ref{lem:subst}),
we know that $\subst{e_2}{e_1}{x} \apprle \subst{t_2}{\lam{}{t_1}}x$,
as required.
\end{itemize}

\item Case $e_0 \apprle (\lam{}{t_0})~\ID$, where $e_0\apprle t_0$: \\
By the induction hypothesis, $\exists t_0'\ e_0\apprle t_0'\ \wedge\ t_0\to t_0'$.
It is easy to see that therefore
$((\lam{}{t_0})~\ID) \stepsone t_0\,\to t_0'$, as required.
\end{itemize}
\end{proof}

Having proved \eqref{eqn:prop}, we can show completeness of the
translation. If we start with a source term $e$ and its translation
$\SB e $, we know from Lemma~\ref{lem:trans} that $e \apprle \SB e $. From
\eqref{eqn:prop}, we know that each step of evaluation of $e$
is mirrored by execution on the target side that preserves $e\apprle t$.
If the evaluation of $e$ diverges, so will the evaluation of $\SB e$.
If the evaluation of $e$ converges on a value $v$, then
the evaluation of $\SB e$ will reach a convergent (by
Lemma~\ref{lem:value-reduction}) term $t$ such that
$v \apprle t$. And by Lemma~\ref{lem:equiv}, $\SB v \approx t$. This demonstrates
completeness.

To show soundness of the translation, we need to show that every
evaluation in the target language corresponds to some evaluation in the
source language. Suppose we have a target-language evaluation
$t\to v'$, where $t = \SB e$, but there is no corresponding
source-language evaluation of $e$. There are three possibilities.
First, the evaluation of $e$ could get stuck. This cannot happen for
this source language because all terms are either values or have a
legal evaluation. Second, the evaluation of $e$ could evaluate to a
value $v$. But then $v$ must correspond to $v'$, because the
target-language evaluation is deterministic. Third, the evaluation of
$e$ might diverge. But then \eqref{eqn:prop} says there is a
divergent target-language evaluation. The determinism of the
target language ensures that cannot happen.
