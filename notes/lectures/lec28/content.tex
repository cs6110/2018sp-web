\handout{28}
\title{Propositions as Types}
\date{20 April 2016}
\maketitle

\section{Intuitionistic Logic and Constructive Mathematics}

We have previously observed that several familiar type judgments $\judge{}e\tau$ of the pure simply-typed $\lambda$-calculus correspond to a tautologies of propositional logic:
\[
\begin{array}{cc}
\mbox{\textit{type judgment}} & \mbox{\textit{propositional tautology}}\\[2pt]\hline\\[-8pt]
\judge{}{I}{\alpha\to\alpha} & P\to P\\
\judge{}{K}{\alpha\to\beta\to\alpha} & P\to (Q\to P)\\
\judge{}{S}{(\alpha\to\beta\to\gamma)\to(\alpha\to\beta)\to(\alpha\to\gamma)} & (P\to Q\to R)\to(P\to Q)\to(P\to R)\\
\end{array}
\]
This is no accident.  It turns out that all derivable type judgments $\judge{}e\tau$ (with the empty environment to the left of the turnstile) give propositional tautologies.  This is because the typing rules of $\lambda^\rightarrow$ correspond exactly to the proof rules of propositional _intuitionistic logic_.

Intuitionistic logic is the basis of _constructive mathematics_.  Constructive mathematics takes a much more conservative view of truth than classical mathematics.  It is concerned less with _truth_ than with _provability_.  Two of its main proponents were Kronecker and Brouwer.  These views generated great controversy in the mathematical world when they first appeared.

In constructive mathematics, not all deductions of classical logic are considered valid.  For example, to prove in classical logic that there exists an object having a certain property, it is enough to assume that no such object exists and derive a contradiction.  Intuitionists would not consider this argument valid.  Intuitionistically, you must actually construct the object and prove that it has the desired property.

Intuitionists do not accept the law of double negation: $P\leftrightarrow \neg\neg P$.  They do believe that $P\to\neg\neg P$, that is, if $P$ is true then it is not false; but they do not believe $\neg\neg P\to P$, that is, even if $P$ is not false, then that does not automatically make it true.

Similarly, intuitionists do not accept the law of the excluded middle $P\vee\neg P$.  In order to prove $P\vee\neg P$, you must prove either $P$ or $\neg P$.  It may well be that neither is provable, in which case the intuitionist would not accept that $P\vee\neg P$.

For intuitionists, the implication $P\to Q$ has a much stronger meaning than merely $\neg P\vee Q$, as in classical logic.  To prove $P\to Q$, one must show how to construct a proof of $Q$ from any given proof of $P$.  So a proof of $P\to Q$ is a (computable) function from proofs of $P$ to proofs of $Q$.  Similarly, to prove $P\wedge Q$, you must prove both $P$ and $Q$; thus a proof of $P\wedge Q$ is a pair consisting of a proof of $P$ and a proof of $Q$.

\subsection{Example}

Here is an example of a proof that would not be accepted by an intuitionist.

\medskip\noindent
\textbf{Theorem}\quad
There exist irrational numbers $a$ and $b$ such that $a^b$ is rational.

\begin{proof}
Either $\sqrt 2^{\sqrt 2}$ is rational or not.  If it is, take $a=b=\sqrt 2$ and we are done.  If it is not, take $a=\sqrt 2^{\sqrt 2}$ and $b=\sqrt 2$; then $a^b=(\sqrt 2^{\sqrt 2})^{\sqrt 2}=\sqrt 2^2=2$, and again we are done.
\end{proof}

Now an intuitionist would not like this, because we have not constructed a definite $a$ and $b$ with the desired property.  We have used the law of the excluded middle, which the intuitionist would regard as cheating.\footnote{It is possible to give a constructive proof using only elementary facts. Hint: consider $\sqrt 2^{\log_2 9}$.}

\section{Syntax}

Syntactically, formulas $\phi,\psi,\ldots$ of intuitionistic logic look the same as their classical counterparts.  At the propositional level, we have propositional variables $P,Q,R,\ldots$ and formulas
\(
\phi &::=& \top \bnf \bot \bnf P \bnf \phi\to\psi \bnf \phi\vee\psi \bnf \phi\wedge\psi \bnf \neg\phi.
\)
We might also add a second-order quantifier $\forall P$ ranging over propositions:
\(
\phi &::=& \cdots \bnf \fa P\phi.
\)

\section{Natural Deduction (Gentzen, 1943)}

Intuitionistic logic uses a sequent calculus to derive the truth of formulas.  Assertions are judgments of the form $\seq\phi 1n\force\phi$, which means that $\phi$ can be derived from the assumptions $\seq\phi 1n$. If $\force\phi$ without assumptions, then $\phi$ is a theorem of intuitionistic logic.  The system is called _natural deduction_.

As we write down the proof rules, it will be clear that they correspond exactly to the typing rules of the pure simply-typed $\lambda$-calculus $\lambda^\to$ (and with quantifiers, System F).  We will show them side by side.  There are generally _introduction_ and _elimination_ rules for each operator.
\[
\begin{array}{ccc}
& \mbox{\textit{intuitionistic logic}} & \mbox{\textit{$\lambda^\to$ or System F type system}}\\ \hline\\
\mbox{(axiom)} & \Gamma,\,\phi\force\phi & \judge{\Gamma,\,x:\tau}x\tau\\[1em]
\mbox{($\to$-intro)}
& \dfrac{\Gamma,\,\phi\force\psi}{\Gamma\force\phi\to\psi}
& \dfrac{\judge{\Gamma,\,x:\sigma}e\tau}{\judge\Gamma{(\lam{x:\sigma}e)}{\sigma\to\tau}}\\[1em]
\mbox{($\to$-elim)}
& \dfrac{\Gamma\force\phi\to\psi \quad \Gamma\force\phi}{\Gamma\force\psi}
& \dfrac{\judge\Gamma{e_0}{\sigma\to\tau} \quad \judge\Gamma{e_1}\sigma}{\judge\Gamma{(e_0~e_1)}\tau}\\[1em]
\mbox{($\wedge$-intro)}
& \dfrac{\Gamma\force\phi\quad\Gamma\force\psi}{\Gamma\force\phi\wedge\psi}
& \dfrac{\judge\Gamma{e_1}\sigma\quad\judge\Gamma{e_2}\tau}{\judge\Gamma{(e_1,e_2)}{\sigma*\tau}}\\[1em]
\mbox{($\wedge$-elim)}
& \dfrac{\Gamma\force\phi\wedge\psi}{\Gamma\force\phi}\quad\dfrac{\Gamma\force\phi\wedge\psi}{\Gamma\force\psi}
& \dfrac{\judge\Gamma e{\sigma*\tau}}{\judge\Gamma{\#1\,e}\sigma}\quad\dfrac{\judge\Gamma e{\sigma*\tau}}{\judge\Gamma{\#2\,e}\tau}\\[1em]
\mbox{($\vee$-intro)}
& \dfrac{\Gamma\force\phi}{\Gamma\force\phi\vee\psi} \quad \dfrac{\Gamma\force\psi}{\Gamma\force\phi\vee\psi}
& \dfrac{\judge\Gamma e\sigma}{\judge\Gamma{"inl"_{\sigma+\tau}}e{\sigma+\tau}} \quad \dfrac{\judge\Gamma e\tau}{\judge\Gamma{"inr"_{\sigma+\tau}}e{\sigma+\tau}}\\[1em]
\mbox{($\vee$-elim)}
& \dfrac{\Gamma\force\phi\vee\psi\quad\Gamma\force\phi\to\chi\quad\Gamma\force\psi\to\chi}{\Gamma\force\chi}
& \dfrac{\judge\Gamma e{\sigma+\tau} \quad \judge\Gamma{e_1}{\sigma\arrow\rho} \quad \judge\Gamma{e_2}{\tau\arrow\rho}}{\judge\Gamma{"case"~e_0~"of"~e_1~|~e_2}\rho}\\[1em]
\mbox{($\forall$-intro)}
& \dfrac{\Gamma,\,P\force\phi}{\Gamma\force\fa P\phi}
& \dfrac
{\Judge{\Delta,\,\alpha}\Gamma e\tau \quad \alpha\notin\FV\Gamma}
{\Judge\Delta\Gamma{(\Lam\alpha e)}{\fa\alpha\tau}}\\[1em]
\mbox{($\forall$-elim)}
& \dfrac{\Gamma\force\fa P\phi}{\Gamma\force\subst\phi\psi P}
& \dfrac
{\Judge\Delta\Gamma e{\fa\alpha\tau} \quad \Delta\force\sigma}
{\Judge\Delta\Gamma{(e~\sigma)}{\subst\tau\sigma\alpha}}
\end{array}
\]
The $\to$-elimination rule is often called _modus ponens_.

\section{The Brouwer--Heyting--Kolmogorov (BHK) Principle}

The fact that propositions in intuitionistic logic correspond to types in our $\lambda$-calculus type systems is known as the _Brouwer--Heyting--Kolmogorov_ (BKH) interpretation or _propositions as types_ principle, and sometimes the _Curry--Howard isomorphism_ (although neither is it due to Curry and Howard nor is it an isomorphism).  The analogy is far reaching:
\begin{center}
\begin{tabular}{ll@{\qquad\qquad}ll}
\multicolumn 2 c{\textit{type theory}} & \multicolumn 2 c{\textit{logic}}\\ \hline
$\tau$ & type & $\phi$ & proposition\\
$\tau$ & inhabited type & $\phi$ & theorem\\
$e$ & well-typed program & $\pi$ & proof\\
$\to$ & function space & $\to$ & implication\\
* & product & $\wedge$ & conjunction\\
+ & sum & $\vee$ & disjunction\\
$\forall$ & type quantifier & $\forall$ & second-order quantifier\\
1 & unit & $\top$ & truth\\
0 & void & $\bot$ & falsity
\end{tabular}
\end{center}
A proof in intuitionistic logic is a construction, which is essentially a program ($\lambda$-term).  Saying that a proposition has an intuitionistic or constructive proof says essentially that the corresponding type is inhabited by a $\lambda$-term.

If we are given a well-typed term in System F or $\lambda^{\arrow}$, then
its proof tree will look exactly like the proof tree for the corresponding
formula in intuitionistic logic.  This means that every well-typed program proves something,
that is, it is a proof in constructive logic.  Conversely, every theorem in constructive
logic corresponds to an inhabited type.  Several automated deduction systems (for example, Nuprl and Coq) are based on this idea.

\section{Theorem Proving and Type Checking}

We have seen that _type inference_ is the process of inferring a type for a given $\lambda$-term.  Under the propositions-as-types principle, this is the same as determining what theorem a given proof proves.  Theorem proving, on the other hand, is going in the opposite direction: Given a formula, does it have a proof?  Equivalently, given a type, is it inhabited?

For example, consider the formula expressing transitivity of implication:
\[
\forall P,Q,R\,.~((P\to Q) \wedge (Q \to R))\ \to\ ( P \to R)
\]
Under the propositions-as-types principle, this is related to the type
\[
\forall\alpha,\beta,\gamma\,.~(\alpha\to\beta)*(\beta\to\gamma)\ \to\ (\alpha\to\gamma).
\]
If we can construct a term of this type, we will have proved the theorem in
intuitionistic logic.  The program
\[
\Lam{\alpha,\beta,\gamma}{\lam {p:(\alpha\to\beta)*(\beta\to\gamma)}{\lam{x:\alpha}{(\#2\,p)~((\#1\,p)~x)}}}
\]
does it.  This is a function that takes a pair of functions as its
argument and returns their composition.
The proof tree that establishes the typing of this function
is essentially an intuitionistic proof of the transitivity of implication.

Here is another example.  Consider the formula
\[
\forall P,Q,R\,.~(P\wedge Q\to R)\ \leftrightarrow\ (P\to Q\to R)
\]
The double implication $\leftrightarrow$ is an abbreviation for the conjunction
of the implications in both directions.  It says that the two formulas on either side
are propositionally equivalent.  The typed expressions corresponding to each side
of the formula above are
\(
\alpha*\beta\to\gamma &\qquad& \alpha\to\beta\to\gamma.
\)
We know that any term of the first type can be converted to one
of the second by _currying_, and we can go in the opposite direction
by _uncurrying_.  The two $\lambda$-terms that convert a function to its
curried form and back constitute a proof of the logical statement.
