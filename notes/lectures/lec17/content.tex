\lecture{17}
\title{Predicate Transformers}
\date{11 March 2016}
\maketitle

\newcommand{\wpre}[2]{\ensuremath{\mathit{wlp}(#1,#2)}}
\newcommand{\hrtrp}[4][~]{\ensuremath{\{ #2 \}{#1}#3{#1}\{ #4 \}}}
\newcommand{\rulename}[1]{{\textsc{#1}}}

\section{Relative Completeness}

In the last lecture, we discussed the issue of completeness---i.e.,
whether it is possible to derive every valid partial correctness
specification using the axioms and rules of Hoare
logic. Unfortunately, if treated as a pure deductive system, Hoare
logic cannot be complete. To see why, consider the following partial
correctness specifications:
%
\[
\hrtrp{\TRUE}{\SKIP}{P} \qquad\qquad \hrtrp{\TRUE}c{\FALSE}
\]
%
The first is valid if and only if the assertion $P$ is valid while the
second is valid if and only if the command $c$ does not halt.

It turns out that the culprit is the weakening rule, which includes
premises involving the validity of implications between the assertions
involved:

\parbox\tl{(weakening)} $\displaystyle\frac{\phi\Imp\phi'\quad\PCA{\phi'}c{\psi'}\quad\psi'\Imp\psi}{\PCA\phi c\psi}$.


Although we cannot have a complete proof system for first-order
formulas, Hoare logic does enjoy the property stated in the following
theorem:
%
\begin{theorem}
%
\( \forall \phi,\psi, c.~\vDash \hrtrp{\phi} c {\psi} \text{ ~~~implies~~~ } \vdash \hrtrp{\phi} c {\psi}.\)
%
\end{theorem}
%
This result, due to Cook (1974), is known as the \emph{relative
  completeness} of Hoare logic. It says that Hoare logic is no more
  incomplete than the language of assertions---i.e., if we had an
  oracle that could decide the validity of assertions, then we could
  decide the validity of partial correctness specifications.

\section{Weakest Preconditions}

Given a program $s$ and a postcondition $\phi$, the weakest property
of the input state that guarantees that $s$ halts in a state
satisfying $\phi$, if it exists, is called the _weakest precondition_
of $s$ and $\phi$ and is denoted $\wp s\phi$.  This says that
\begin{itemize}
\item
$\wp s\phi$ implies that $s$ terminates in a state satisfying $\phi$ ($\wp s\phi$ is a precondition of $s$ and $\phi$),
\item
if $\psi$ is any other condition that implies that $s$ terminates in a state satisfying $\phi$, then $\psi\Rightarrow\wp s\phi$ ($\wp s\phi$ is the _weakest precondition_ of $s$ and $\phi$).
\end{itemize}

As in the $\lambda$-calculus, juxtaposition represents function
application, so $\mathsf{wp}$ can be viewed as a higher-order function
that takes a program $s$ and a postcondition $\phi$ and returns the
weakest precondition of $s$ and $\phi$.  The function $\mathsf{wp}$
can also be viewed as taking a program and returning a function that
maps postconditions to preconditions. For this reason, axiomatic
semantics is sometimes known as _predicate transformer semantics_.

\section{Weakest Liberal Preconditions}

Cook's proof of relative completeness depends on the notion of
\emph{weakest liberal preconditions}. Given a command $c$ and a
postcondition $\psi$ the weakest liberal precondition is the weakest
assertion $\phi$ such that \(\hrtrp{\phi}{c}{\psi}\) is a valid triple. Here,
``weakest'' means that any other valid precondition implies $\phi$. That
is, $\phi$ most accurately describes input states for which $c$ either
does not terminate or ends up in a state satisfying $\psi$.

Formally, an assertion $\phi$ is a weakest liberal precondition of $c$
and $\psi$ if:
%
\[
\forall \sigma, I.~ \sigma \vDash_I \phi \iff (\Tr{C}{c}~\sigma) 
\text{ undefined } \vee (\Tr{C}{c} \sigma) \vDash_I \psi
\]
%
We write $\wpre{c}{\psi}$ for the weakest liberal precondition of command
$c$ and postcondition $\psi$. From left-to-right, the formula above
states that $\wpre{c}{\psi}$ is a valid precondition: $\vDash
\hrtrp{\phi}{c}{\psi}$. The right-to-left implication says it is the weakest
valid precondition: if another assertion $R$ satisfies $\vDash
\hrtrp{R}{c}{\psi}$, then $R$ implies $\phi$.  It can be shown that weakest
liberal preconditions are unique modulo equivalence.

We can calculate the weakest liberal precondition of a command as
follows:
%
\[
\begin{array}{rcl} 
\wpre{\SKIP}{\phi} & = & \phi\\
\wpre{(x := a}{\phi} &=& \phi[a/x]\\
\wpre{(c_1;c_2)}{\phi} &=& \wpre{c_1}{\wpre{c_2}{\phi}}\\
\wpre{\cond{b}{c_1}{c_2}}{\phi} &=& (b \implies \wpre{c_1}{\phi}) \wedge (\neg b \implies \wpre{c_2}{\phi})\\
\end{array}
\]
%
The definition of $\wpre{\while{b}{c}}{\phi}$ is slightly more
complicated---it encodes the weakest liberal precondition for each
iteration of the loop. To give the intuition, first define the weakest
liberal precondition for a loop that termintes in $i$ steps as
follows:
%
\[
\begin{array}{rcl}
F_0(\phi) & = & \TRUE\\
F_{i+1}(\phi) & = & (\neg b \implies \phi) \wedge (b \implies \wpre{c}{F_i(\phi)})
\end{array}
\]
%
We can then express the weakest liberal precondition using an
infinitary conjunction:
%
\[
\wpre{\while{b}{c}}{\phi} = \bigwedge_i F_i(\phi)
\]
%
See Winskel Chapter 7 for the details of how to encode the weakest
liberal precondition for a while loop as an ordinary assertion.
 
To check that our definition is correct, we can prove (how?) that it
yields a valid partial correctness specification:
%
\begin{lemma}\label{lem:wlp1}
\[
\forall c, \psi.~ \vDash \hrtrp{\wpre{c}{\psi}}{c}{\psi}~~\text{and}~~
\forall \rho.~\vDash \hrtrp{\rho}{c}{\psi} \text{ implies } (\rho \implies \wpre{c}{\psi})
\]
\end{lemma}
%
It is not hard to prove that it also yields a provable specification:
%
\begin{lemma}\label{lem:wlp2}
\[
\forall c, \psi.~ \vdash \hrtrp{\wpre{c}{\psi}}{c}{\psi}
\]
\end{lemma}
%
\noindent Relative completeness follows by a simple argument:
%
\begin{proof}[Proof Sketch] Let $c$ be a command and let $P$ and $\psi$
  be assertions such that the partial correctness specification
  $\hrtrp{P}{c}{\psi}$ is valid. By Lemma~\ref{lem:wlp1} we have $\vDash
  P \implies \wpre{c}{\psi}$. By Lemma~\ref{lem:wlp2} we have
  $\vdash\hrtrp{\wpre{c}{\psi}}{c}{\psi}$. We conclude $\vdash
  \hrtrp{P}{c}{\psi}$ using weakening.
\end{proof}
