\lecture{1}
\title{Introduction}
\date{25 August 2006}
\maketitle

\section{Notation Used in This Course}

Our meta-system consists of ordinary naive set theory based on the Zermelo--Fraenkel axioms with the axiom of choice (ZFC).  We will assume thorough familiarity with the set-theoretic account of basic constructs such as Cartesian product $\times$, functions, relations, partial orders, equivalence relations, and the natural numbers $\naturals$.


Functions $f:A\to B$

Functional composition 

Relational composition

diagrammatic order $R;S = S\circ R$

When studying logical formalisms, we often need to make meta-statements \emph{about} the system we are studying that on paper look very similar to statements \emph{in} the object of study.  This is the classical \emph{use vs.~mention} conflict.  For example, our programming language may contain a conditional test $\ifthenelse{b}{p}{q}$, whose semantics we will need to define using a meta-conditional
\begin{align}
\sem{\ifthenelse{b}{p}{q}}\sigma &\definedas \dcond{\sem b\sigma}{\sem p\sigma}{\sem q\sigma}\label{eqn:ambig1}\\
&= \begin{cases}
\sem p\sigma & \text{if } \sem b\sigma = \true\\
\sem q\sigma & \text{if } \sem b\sigma = \false
\end{cases}\label{eqn:ambig2}
\end{align}
(don't worry about the undefined notation $\sem p\sigma$ for right now).  In this case we can avoid confusion by preferring the notation \eqref{eqn:ambig2} over \eqref{eqn:ambig1}, but sometimes ambiguity is unavoidable when the same symbols are used at both levels.  We will do our best to avoid confusion wherever possible by using different symbols and fonts.  In particular, we will tend to use a \textsf{sans-serif font} for constructs in the programming language under study and the normal text font for meta-statements, as in the example above.

